\documentclass[a4paper, 12pt]{article}
%\usepackage{portuguese}{babel}
\usepackage{subfiles}
\usepackage{forloop}
\usepackage{arydshln}
\usepackage{fancyhdr}
\usepackage{graphicx}
\grahicspath{ {./assets/} }

\title {Desenvolvimento de Sistemas de Software}

\author{\textbf{Grupo 31} \\
        62608 - Marco Sousa \\
        93198 - Mariana Marques \\
        93271 - José Malheiro \\
        94269 - Miguel Fernandes}

\begin{document}
\maketitle

\section{Modelação de Domínio}
% adicionar o modelo de Domínio
% adicionar descrição do modelo
\subsection{Entidades}

\begin{itemize}
        \item[Cliente]{} % NIF
        \item[Colaborador] {} %% prazo máximo de reparação ; Preço
        \item[Equipamento] {} %% Código de registo
        \item[Orçamento] {}
        \item[Reparação] {} % Reparação normal - Passo da Reparação, tempo, peças/ material utilizadas - calculo do preço através do material (ter uma tabela)
                            % Reparação expresso - Preço fixo; serviços disponíveis?
        \item[Forma de Contacto] {}
\end{itemize}
% Cliente

% Orçamento

% Equipamento



%%% 
% Colaborador
%% Funcionário do Balcão
%% Técnico
%% Gestor
% Formas de Contacto
%% SMS
%% Email

% descrição das entidades
\subsection{Diagrama de Modelo de Domínio} % diagrama ??

\section{Modelação dos Requisitos Funcionais}
% O que o sistema deve fazer
Cenários foram identificados pela equipa docente.
% adicionar o diagrama de use case
A partir dos cenários foram identificados os seguintes \textit{use cases}:
\begin{itemize}
        \item Pedir orçamento - \ref{pedir_orcamento}
        \item Fazer orçamento - \ref{fazer_orcamento}
        \item Confirmar orçamento - \ref{confirmar_orcamento}
        \item Arquivar orçamento - \ref{arquivar_orcamento}
        \item Realizar reparação - \ref{realizar_rep}
        \item Entregar equipamento - \ref{entregar_equipamento}
        \item Pedir reparação expresso - \ref{pedir_rep_xpress}
        \item Registar equipamento - \ref{registar_equipamento}
        \item Registar cliente - \ref{registar_cliente}
        \item Listar resumida do técnico - \ref{listagem_tecnico_resumida}
        \item Listar detalhada do técnico - \ref{listagem_tecnico_detalhada}
        \item Listar funcionário do balcão - \ref{listagem_func_balcao}
\end{itemize}
\subsection{Modelo de Use Cases}

\subsection{Especificação Use Cases}

\subsubsection{} \label{pedir_orcamento}
\subfile{use_cases/usecase_pedir-orcamento.tex}

\subsubsection{} \label{fazer_orcamento}
\subfile{use_cases/usecase_fazer-orcamento.tex}

\subsubsection{} \label{confirmar_orcamento}
\subfile{use_cases/usecase_confirmar-orcamento.tex}

\subsubsection{} \label{arquivar_orcamento}
\subfile{use_cases/usecase_arquivar-orcamento.tex}

\subsubsection{} \label{realizar_rep}
\subfile{use_cases/usecase_realizar-reparacao.tex}

\subsubsection{} \label{entregar_equipamento}
\subfile{use_cases/usecase_entregar-equipamento.tex}

\subsubsection{} \label{pedir_rep_xpress}
\subfile{use_cases/usecase_pedir-reparacao-expresso.tex}

\subsubsection{} \label{registar_equipamento}
\subfile{use_cases/usecase_registar-equipamento.tex}

\subsubsection{} \label{registar_cliente}
\subfile{use_cases/usecase_registar-cliente.tex}

\subsubsection{} \label{listagem_tecnico_resumida}
\subfile{use_cases/usecase_listar-tecnico-resumida.tex}

\subsubsection{} \label{listagem_tecnico_detalhada}
\subfile{use_cases/usecase_listar-tecnico-detalhada.tex}

\subsubsection{} \label{listagem_func_balcao}
\subfile{use_cases/usecase_listar-func-balcao.tex}
\end{document}