\documentclass[a4paper,12pt]{scrreprt}
    %% Used for changing geometry of the page
    %% Cover page text cannot overlay cover sketching/style 
    %% https://ctan.org/pkg/geometry?lang=en
\usepackage{geometry}
    %% Changes language of some packages protocols
    %% e.g., when captioning images: Figure 1. -> Figura 1.
    %% https://ctan.org/pkg/babel?lang=en
\usepackage[portuguese]{babel}
    %% Used for special fonts
    %% Cannot be compiled with pdflatex
    %% https://ctan.org/pkg/fontspec?lang=en
\usepackage{fontspec}
    %% Arial FONT
    \setmainfont{Arial}
\usepackage{subfiles}
    %% More colors and color options
    %% https://ctan.org/pkg/xcolor?lang=en
    %% https://ctan.org/pkg/colortbl?lang=en
\usepackage{xcolor,colortbl}
    %% More tabular options, like dashed/dotted lines
    %% https://ctan.org/pkg/arydshln?lang=en
\usepackage{arydshln}
    %% List of acronyms
    %% https://ctan.org/pkg/nomencl?lang=en
\usepackage[intoc]{nomencl}
    %% Must be called to init nomencl environment  
    \makenomenclature
    %% More images options/settings
    %% https://ctan.org/pkg/graphicx?lang=en
\usepackage{graphics}
    %% Defining subdirectories to image path enviornment
    %% \graphicspath{{sub1}{sub2}...{subN}}
    \graphicspath{{images}}
    
    %% used to handle cross-referencing commands in LaTeX to produce hypertext links in the document
    %% https://ctan.org/pkg/hyperref?lang=en
\usepackage{hyperref}
    %% math environments
    %% https://ctan.org/pkg/amsmath?lang=en

    %% settings
    \hypersetup{
        colorlinks,
        citecolor=black,
        filecolor=black,
        linkcolor=black,
        urlcolor=black
    }

\usepackage{amsmath}
    %% Defining backgrouns, used to make the cover
    %% https://ctan.org/pkg/background?lang=en
\usepackage[some]{background}
    %% Used to make drawings or complex graphics
    %% http://pgf.sourceforge.net/pgf_CVS.pdf
\usepackage{tikz}
    %% Tikz library to point operations ((x1,y1) + (x2,y2))
    \usetikzlibrary{calc}
\usepackage{lscape}
%% Defining sfdefault font and default font for document
\renewcommand{\familydefault}{\sfdefault}


%% Costume made cover 
%% From there you can use \makecover command to build the cover
%% Blue cover color
\definecolor{titlepagecolor}{RGB}{54,95,145}

%==========================================================================
% COLORED BAR ON THE LEFT SIDE
%==========================================================================

\backgroundsetup{
    scale=1,
    angle=0,
    opacity=1,
    contents={
            \begin{tikzpicture}[remember picture,overlay]
                \path [fill=titlepagecolor]
                (current page.north west) -- ($(current page.north west) + (5,0)$)
                -- ($(current page.south west) + (5,0)$)-- (current page.south west);
                \node[color=white] at ($(current page.south west) + (3,4)$) {\bfseries {\fontsize{50}{60} \textsf{DSS}}};
                %\node[color=titlepagecolor] at ($(current page.south west) + (5.8,4)$) {\bfseries {\fontsize{120}{60} \textsf{4}}};
            \end{tikzpicture}
        }
}

%==========================================================================
% TITLE PAGE INFO
%==========================================================================

%% Changes values in this field to show information in the cover and back cover about your team/project


%% TITLE
\title{Sistema de apoio a departamento técnico}

%% AUTHORS
\author{
    \begin{tabular} { c c }
        \includegraphics[scale=0.2]{author/marco.jpg} & \includegraphics[scale=0.2]{author/marco.jpg} \\
        62608 - Marco Sousa                           & 93198 - Mariana Marques                       \\
        %\hline                                                                                        \\
        \includegraphics[scale=0.2]{author/ze.jpg} & \includegraphics[scale=0.2]{author/miguel.jpg} \\
        93271 - José Malheiro                         & 94269 - Miguel Fernandes
    \end{tabular}
}

%% Date

\date{\today}

%% Course
\newcommand{\Course}{Licenciatura em Engenharia Informática}

%% Department
\newcommand{\Department}{Escola de Engenharia}

%% UniName
\newcommand{\UniName}{Universidade do Minho}

\newcommand{\UcName}{Desenvolvimento de Sistemas de Software}

\newcommand{\GroupId}{Grupo 31}

%% UniPic
\newcommand{\UniPic}{\includegraphics[scale=0.09]{uminho.png}}

%% University 
\newcommand{\University}{
    \begin{flushleft}
        \UniPic
    \end{flushleft}
    \textcolor{gray}{\small\textbf{\textsf{\UniName}}}\par
    \textcolor{gray!80!white}{\small{\textsf{\Department}}}\par
    \textcolor{gray!70!white}{\small{\textsf{\Course}}}
}

%% UC
\newcommand{\UC}{
    \begin{flushleft}
        \par\textcolor{titlepagecolor}{  \LARGE\textbf{\textsf{Unidade Curricular de \\ \UcName}}}
    \end{flushleft}
}

%% School Year
\newcommand{\SchoolYear}{
    \small{\textsf{Ano Letivo de 2021/2022}}}


%% Define new command to show title, author and date
\makeatletter
\let\Title\@title
\let\Author\@author
\let\Date\@date
\makeatother

%==========================================================================
% CLASSIFICATION SECTION 
%==========================================================================

%% School Year
\newcommand{\ReceptionDate}{}
%% Responsible
\newcommand{\Responsible}{}
%% Evaluation
\newcommand{\Evaluation}{}
%% Observations
\newcommand{\Observations}{}





%% MAKETEMPLATE
\newcommand{\makecover}{

    %==========================================================================
    % BEGIN COVER PAGE 
    %==========================================================================

    %% Removes page number on footer
    \thispagestyle{empty}

    %% No indentation 
    \setlength{\parindent}{0em}

    %% Put Background defined on \backgroundsetup, in this page
    \BgThispage

    %% Changing geometry to prevent overlay with text
    %% At the end of back cover, geometry is default with \restoregeometry
    \newgeometry{top=3.5cm,left=6cm,right=3cm,bottom=2cm}

    %% builds university info defined previously
    \University
    \vspace{1cm}
    %% builds curricular unity info defined previously
    \UC
    %% builds school year info defined previously
    \SchoolYear

    \vspace*{4cm}
    %% bigger space (i think its the default one) between paragraphs 
    \setlength{\parskip}{1em}

    %% builds title info defined previously
    \par\textbf{\textsf{\huge\Title}}
    \par\textbf{\GroupId}
    \vspace{1cm}
    %% builds author(s) info defined previously
    \par\begin{center}
        \Author
    \end{center}

    \vspace{0.5cm}

    %% builds date info defined previously
    \par\Date
    \restoregeometry
    \pagebreak

    %==========================================================================
    % END COVER PAGE 
    %==========================================================================

    %==========================================================================
    % BEGIN BACK COVER PAGE 
    %==========================================================================

    %% Removes page number on footer
    % \thispagestyle{empty}

    % % Changing look of lines in tabular environment 
    % % Dashed -> dotted 
    % %% length of dashes
    % \setlength\dashlinedash{0.3pt}
    % %% space between dashes
    % \setlength\dashlinegap{1.5pt}
    % %% width of dashes
    % \setlength\arrayrulewidth{1.1pt}


    % %% This values can be changed in the preamble
    % \begin{flushright}
    %     \begin{tabular}{ :p{4cm}:p{4cm}: }
    %         \hdashline
    %         Data de Receção & \ReceptionDate \\ [2ex]
    %         \hdashline
    %         Responsável     & \Responsible   \\ [2ex]
    %         \hdashline
    %         Avalição        & \Evaluation    \\ [2ex]
    %         \hdashline
    %         Observações     & \Observations  \\ [7ex]
    %         \hdashline
    %     \end{tabular}
    % \end{flushright}


    % \vspace{10cm}
    % \begin{flushleft}

    %     %% builds title info defined previously
    %     \par\textbf{\textsf{\huge\Title}}
    %     \vspace{1cm}
    %     %% builds author info defined previously
    %     \par\Author

    %     \vspace{0.5cm}

    %     %% builds date info defined previously
    %     \par\Date
    % \end{flushleft}

    % \pagebreak
    %==========================================================================
    % END BACK COVER PAGE 
    %==========================================================================
}


\graphicspath{ {./assets/} }

% TODO
% ABSTRACT - inclui objetivos e descrição MARCO
% abordar motivo de criar passos e material
% considerações finais

\begin{document}

\pagenumbering{gobble}

% builds the cover
\makecover

%==========================================================================
% BEGIN ABSTRACT PAGE
%==========================================================================



%% Abstract name: \Large font size, flushed left and paragraph skip before abstract content
\renewenvironment{abstract}
{\par\noindent\textbf{\Large\abstractname}\par\bigskip}
{}

\begin{flushleft}
    \begin{abstract}
        %=============
        % How to build an abstract
        % https://users.ece.cmu.edu/~koopman/essays/abstract.html
        %=============
        A utilização de modelos auxilia a compreensão do problema, simplifica a comunicação de ideias e documenta as decisões tomadas durante o desenvolvimento.
        
        Um centro de reparações encontra problemas diários que limitam a sua capacidade de resposta,
        diminuindo a eficiência dos colaboradores e impactando no volume de trabalho concretizado.
        
        Através da análise dos cenários de utilização, foi possível efetuar uma modelação do domínio.
        Posteriormente, conseguiu-se realizar a modelação dos requisitos funcionais.
        Assim, foi desenvolvido um \textbf{Modelo de Domínio} e um \textbf{Diagrama de \textit{Use Case}},
        assim como as respetivas descrições dos mesmos.
        
        A estratégia utilizada permitiu seguir uma linha de raciocínio estruturada e ao utilizar uma linguagem de modelação unificada obteve-se
        um conjunto de diagramas que são facilmente interpretáveis entre pares.
        
        \par \textbf{Área de Aplicação}: Análise de Requisitos
        \par \textbf{Palavras-Chave}: \textit{Unified Modeling Language} (UML), \textit{Visual Paradigm}, Modelação de Domínio, Modelação de Requisitos Funcionais, Diagrama de \textit{Use Case}, Modelo de Domínio, Análise de Projeto, Conceção de Projeto, Análise de Requisitos
    \end{abstract}
\end{flushleft}


\pagebreak

%==========================================================================
% END ABSTRACT PAGE 
%==========================================================================

%==========================================================================
% BEGIN INDEXES PAGES
%==========================================================================

%% Changes table of content name
%% Portuguese babel default : "Conteúdo"
%% Personally I prefer "índice"
\renewcommand{\contentsname}{Índice}

\tableofcontents

\pagebreak

\listoffigures

\pagebreak

%\listoftables

%\pagebreak

%==========================================================================
% END INDEXES PAGES 
%==========================================================================


%==========================================================================
% BEGIN INTRODUCTION
%==========================================================================

%% Starting page numbering here
\pagenumbering{arabic}

\chapter{Introdução}
O presente relatório foi desenvolvido no âmbito da Unidade Curricular (UC) de Desenvolvimento de Sistemas de Software (DSS),
tendo como principal objetivo apresentar uma possível conceção de um Sistema de Gestão para Centros de Reparação de Equipamentos Eletrónicos, doravante designado \textit{aplicação}.

\section{Contextualização}
A procura por um serviço de reparação, compreende um atendimento célere e garantia de acompanhamento ao longo de todo o processo.
Para que tal seja possível, foi proposto pela equipa docente o desenvolvimento de uma aplicação que permita essa gestão.

Desta forma, foi apresentado um conjunto de possíveis cenários de utilização que a aplicação deverá ser capaz de suportar,
dos quais se efetuou um levantamento de requisitos através de estratégias de modelação lecionadas na UC de DSS.

\section{Breve descrição do enunciado proposto}

Tal como referido, o enunciado propõe a conceção e implementação de uma aplicação e apresenta os cenários de utilização que deverão ser
suportados por esta.

Para a fase a que este relatório se refere - \textbf{análise}, inclui-se 
(i) a análise do domínio do problema, através da modelação de um modelo de domínio e 
(ii) de requisitos funcionais, onde se utlizará a modelação de requisitos funcionais, com uma visão orientada aos \textit{use cases}.

\section{Objectivos}
Pretende-se efetuar uma abordagem ao problema utilizando uma estratégia estruturada e sistemática, concretizando a modelação do problema.
Assim, é possível obter um \textbf{modelo de domínio} que fornece uma \textit{framework} conceptual para raciocinar sobre o problema.
Aqui, é capturado as \textbf{Entidades} do problema e os \textbf{Relacionamentos} entre elas.

Após esta primeira etapa, procura-se analisar os requisitos, em particular os funcionais, pretendendo descrever o que o sistema deve fazer.

Para além destes, o grupo pretende, ainda, desenvolver competências técnicas e conceptuais no desenvolvimento de modelos e diagramas,
por forma a se preparar para o mercado de trabalho de \underline{Engenharia de Software}.

\section{Estrutura do Relatório}
Atendendo à estratégia definida, o grupo começará por apresentar a \textbf{Modelação de Domínio} - \ref{modelacao_dominio},
onde se pode encontrar as várias \textbf{Entidades} capturadas, assim como o \textbf{Relacionamento} entre elas.
Posteriormente, apresenta-se a \textbf{Modelação dos Requisitos Funcionais} - \ref{modelacao_req_funcionais}, 
onde se procurou descrever o que o sistema deve fazer, utilizando uma visão orientada aos \textit{Use Cases}.
Por fim, foi efetuada uma breve análise crítica do trabalho desenvolvido nesta fase.

%==========================================================================
% END INTRODUCTION
%==========================================================================

\chapter{Modelação de Domínio} \label{modelacao_dominio}
Numa tentativa de capturar os elementos intervenientes do problema, as entidades e o relacionamento entre eles foi necessária
a criação de uma visão estática do mesmo, \textit{i.e.} um modelo de domínio. 

Com o auxílio deste será formulada uma \textit{framework} conceptual do sistema, permitindo raciocinar sobre o problema e estabelecer o 
vocabulário a ser usado no decorrer do projeto.

\section{Descrição do problema}
O centro de reparações oferece dois tipos de intervenções:
\begin{itemize}
    \item[Normal]{Uma intervenção de duração variável, em que o equipamento é reparado após a aprovação do cliente face a um orçamento criado.}
    \item[Expresso]{Conjunto de serviços limitado, com preço fixo e 
                    que apenas é aceite mediante disponiblidade de tempo para realização imediata.
                   }
\end{itemize}

Para desenvolver o orçamento é necessário realizar uma avaliação do equipamento e definir um plano de trabalhos. 
Este consiste na sequência de passos da reparação, cada um apontando o custo, a quantidade de material necessária e o tempo 
previsto da reparação.

Os colaboradores são os principais atores do sistema. 
O funcionário de balcão identifica o(s) cliente(s) e o(s) respetivo(s) equipamento(s). Posteriormente, efetua o pedido de orçamento.
Assume-se que um cliente pode ter um ou mais equipamentos.
Sendo que o equipamento pode avariar mais do que uma vez, torna-se relevante a sua entidade ser reutilizada entre cada reparação.
Por isso, um equipamento tem um histórico de reparações, assim como histórico de orçamentos.

O técnico é o responsável pela especificação do orçamento e realização da reparação.
Por uma questão de otimização de tempo e recursos, um técnico pode estar associado a mais do que uma reparação,
assim como uma reparação pode ser executada por mais do que um técnico 
(por exemplo, quando um técnico termina o seu turno, outro pode pedir ao sistema uma reparação pendente e dar-lhe continuidade).
Para o desenvolvimento de um orçamento, o técnico constrói um novo plano de trabalhos.
Este, por sua vez, é constituído por um conjunto de passos de reparação.
Para tornar o sistema mais flexível, optou-se por associar uma categoria de material ao passo,
permitindo que o mesmo \textbf{passo de reparação} possa ser reutilizado em mais que um plano de trabalhos.
Ainda com o mesmo propósito, optou-se por associar qualquer tipo de reparação à entidade \textbf{Reparação},
sendo que esta contém um plano de trabalhos.
Esta estratégia permite que até a reparação expresso (que tem características particulares - serviço fixo),
possa ser caracterizada por um plano de trabalhos.

Atendendo que a reparação gasta \textbf{Material}, optou-se por acrescentar a \textbf{Quantidade} de material em \textit{stock}.
Esta alteração permitiu calcular quais as reparações que podem avançar,
bem como o material que é necessário encomendar (ver \ref{encomendar_material}).

Por último, o gestor terá um papel de administração, avaliando os desempenhos dos outros elementos e
recrutando novos colaboradores (tendo de os adicionar ao sistema).

Neste sentido, como resposta à necessidade de modelar o problema e com base na análise do enunciado proposto foram identificadas as 
principais entidades (\ref{ent}) que fazem parte do sistema, assim como os relacionamentos entre elas. 

\section{Entidades}\label{ent}

\subsection{Cliente}\label{ent_cliente}
O cliente é a entidade que possui o(s) equipamento(s) que necessita(m) de intervenção técnica, cada cliente é identificado pelo 
seu Número de Identificação Fiscal (NIF).

\subsection{Colaborador} \label{ent_colaborador}
O Colaborador é a entidade relativa a todos os elementos que interagem diretamente com o sistema. 
Numa forma geral, engloba todos os trabalhadores que populam o \textbf{Centro de Reparação} e, após autenticados, executam os
serviços pedidos - Uma entidade abstrata que ramifica-se em sub-entidades(trabalhadores) mais específicas.

Desde a inserção do equipamento a ser reparado no sistema até à sua eventual entrega ao cliente, o processo conta com participação
do(s) \textbf{Funcionário(s) de Balcão} e do(s) \textbf{Colaboradore(s) Especializado(s)}.

\subsubsection{Funcionário de Balcão} \label{ent_func_balcao}
Corresponde à entidade responsável pelo início e o término de uma reparação - o registo inicial e entrega do equipamento ao cliente, respetivamente.  

\subsubsection{Colaborador Especializado} \label{ent_colab_especializado}
Esta entidade compreende os colaboradores com competências técnicas especializadas, divergindo na sua autoridade com o Funcionário de Balcão. 
Estes possuem a capacidade de alterar o que considerarem necessário no Centro de Reparação.

Desta forma, foi definido os seguintes tipos:
\begin{itemize}
    \item[\textbf{Técnico}]{Responsável por fazer o orçamento e realizar a reparação}
    \item[\textbf{Gestor}]{Tem permissões de administração sobre todo o sistema.}
\end{itemize}

\subsection{Equipamento} \label{ent_equipamento}
Corresponde a uma entidade que representa um objeto físico que irá entrar no sistema com a perspetiva de obter um orçamento e,
eventualmente, originar uma reparação.
Esta entidade é identificada por um código de registo e marca.
Optou-se por um identificador duplo - \{marca, código de registo\}, 
por forma a garantir que cada equipamento é facilmente identificado sem repetições.
O código de registo é o número de série.

\subsection{Orçamento} \label{ent_orcamento}
O orçamento é a entidade associada a um plano de trabalhos estabelecido pelo técnico e ao prazo máximo de execução da reparação.

\subsection{Reparação} \label{ent_reparacao}
Representa um serviço disponibilizado no sistema. Cada reparação contém um plano de trabalhos e,
mediante o tipo de serviço, apresenta características distintas.
Mediante os cenários apresentados, foram definidos dois tipos: \textit{Reparação expresso} e \textit{Reparação normal}.

\subsubsection{Reparação expresso} \label{ent_reparacao-expresso}
Define um tipo de reparação pré-definida.
Tem associado um plano de trabalhos fixo que, por sua vez, origina um custo e tempo de realização fixo.

\subsubsection{Reparação normal} \label{ent_reparacao-normal}
Corresponde a uma reparação personalizada que contém um plano de trabalhos definido.
Consequentemente, o preço e tempo varia entre cada uma.

\subsection{Forma de Contacto} \label{ent_formas-contacto}
A forma de contacto permite uma comunicação com o cliente, sendo registada a data e hora e colaborador que a efetuou.
Para tal, as opções existentes são o SMS e o Email.

\subsection{Plano de Trabalhos} \label{ent_plano-de-trabalhos}
O plano de trabalhos é divido em passos e sub-passos da reparação, onde cada um consome tempo e utiliza material. 
Esta definição permite obter o número total de horas de trabalho e o custo das peças utilizadas. 
Cada passo de reparação utiliza uma determinada categoria de material, possibilitando obter uma lista de material que pode ser utilizado.

\subsection{Material} \label{ent_material}
O material é uma entidade fundamental tanto para a previsão do orçamento como para a própria reparação.
Este é identificado pela sua referência, está associado a um custo e está contido na categoria respetiva.

\section{Diagrama de Modelo de Domínio}

\begin{figure}[!ht]
    \centering
    \includegraphics[scale=0.40]{Modelo.jpg}
    \caption{Modelo de Domínio}
\end{figure}

\chapter{Modelação dos Requisitos Funcionais} \label{modelacao_req_funcionais}

A partir dos cenários (ver \ref{cenarios}) identificados pela equipa docente, 
foram identificados os seguintes \textit{use cases}:

\begin{itemize}
    \item Criar reparação expresso - \ref{criar_rep_expresso}
    \item Pedir orçamento - \ref{pedir_orcamento}
    \item Fazer orçamento - \ref{fazer_orcamento}
    \item Criar passo da Reparação - \ref{criar_passo_rep}
    \item Confirmar orçamento - \ref{confirmar_orcamento}
    \item Arquivar orçamento - \ref{arquivar_orcamento}
    \item Realizar reparação - \ref{realizar_rep}
    \item Realizar passo de reparação - \ref{realizar_passo_rep}
    \item Adicionar material - \ref{adicionar_material}
    \item Encomendar material - \ref{encomendar_material}
    \item Receber material - \ref{receber_material}
    \item Entregar equipamento - \ref{entregar_equipamento}
    \item Dar baixo do equipamento - \ref{dar_baixa_equipamento}
    \item Pedir reparação expresso - \ref{pedir_rep_xpress}
    \item Registar equipamento - \ref{registar_equipamento}
    \item Registar cliente - \ref{registar_cliente}
    \item Registar Colaborador - \ref{registar_colab}
    \item Autenticar Colaborador- \ref{autenticar_colab}
    \item Listar resumida do técnico - \ref{listagem_tecnico_resumida}
    \item Listar detalhada do técnico - \ref{listagem_tecnico_detalhada}
    \item Listar funcionário do balcão - \ref{listagem_func_balcao}
\end{itemize}
\section{Modelo de Use Cases}

\begin{figure}[!ht]
    \centering
    \includegraphics[scale=0.45]{dss-usecase.jpg}
    \caption{Diagrama de \textit{Use Case}}
\end{figure}

\section{Especificação Use Cases}

\subsection{} \label{criar_rep_expresso}
\subfile{use_cases/usecase_criar-rep-xpress.tex}
\noindent{\rule{\textwidth}{0.4pt}

\subsection{} \label{pedir_orcamento}
\subfile{use_cases/usecase_pedir-orcamento.tex}
\noindent{\rule{\textwidth}{0.4pt}

\subsection{} \label{fazer_orcamento}
\subfile{use_cases/usecase_fazer-orcamento.tex}
\noindent{\rule{\textwidth}{0.4pt}

\subsection{} \label{criar_passo_rep}
\subfile{use_cases/usecase_criar-passo.tex}
\noindent{\rule{\textwidth}{0.4pt}

\subsection{} \label{confirmar_orcamento}
\subfile{use_cases/usecase_confirmar-orcamento.tex}
\noindent{\rule{\textwidth}{0.4pt}

\subsection{} \label{arquivar_orcamento}
\subfile{use_cases/usecase_arquivar-orcamento.tex}
\noindent{\rule{\textwidth}{0.4pt}

\subsection{} \label{realizar_rep}
\subfile{use_cases/usecase_realizar-reparacao.tex}
\noindent{\rule{\textwidth}{0.4pt}

\subsection{} \label{realizar_passo_rep}
\subfile{use_cases/usecase_realizar-passo-reparacao.tex}
\noindent{\rule{\textwidth}{0.4pt}

\subsection{} \label{adicionar_material}
\subfile{use_cases/usecase_adicionar-material.tex}
\noindent{\rule{\textwidth}{0.4pt}

\subsection{} \label{encomendar_material}
\subfile{use_cases/usecase_encomendar-material.tex}
\noindent{\rule{\textwidth}{0.4pt}

\subsection{} \label{receber_material}
\subfile{use_cases/usecase_receber-material.tex}
\noindent{\rule{\textwidth}{0.4pt}

\subsection{} \label{entregar_equipamento}
\subfile{use_cases/usecase_entregar-equipamento.tex}
\noindent{\rule{\textwidth}{0.4pt}

\subsection{} \label{dar_baixa_equipamento}
\subfile{use_cases/usecase_dar-baixa.tex}
\noindent{\rule{\textwidth}{0.4pt}

\subsection{} \label{pedir_rep_xpress}
\subfile{use_cases/usecase_pedir-reparacao-expresso.tex}
\noindent{\rule{\textwidth}{0.4pt}

\subsection{} \label{registar_equipamento}
\subfile{use_cases/usecase_registar-equipamento.tex}
\noindent{\rule{\textwidth}{0.4pt}

\subsection{} \label{registar_cliente}
\subfile{use_cases/usecase_registar-cliente.tex}
\noindent{\rule{\textwidth}{0.4pt}

\subsection{} \label{registar_colab}
\subfile{use_cases/usecase_registar-colaborador.tex}
\noindent{\rule{\textwidth}{0.4pt}

\subsection{} \label{autenticar_colab}
\subfile{use_cases/usecase_autenticar-colaborador.tex}
\noindent{\rule{\textwidth}{0.4pt}

\subsection{} \label{listagem_tecnico_resumida}
\subfile{use_cases/usecase_listar-tecnico-resumida.tex}
\noindent{\rule{\textwidth}{0.4pt}

\subsection{} \label{listagem_tecnico_detalhada}
\subfile{use_cases/usecase_listar-tecnico-detalhada.tex}
\noindent{\rule{\textwidth}{0.4pt}

\subsection{} \label{listagem_func_balcao}
\subfile{use_cases/usecase_listar-func-balcao.tex}
\noindent{\rule{\textwidth}{0.4pt}

\chapter{Considerações Finais}
Ao longo da realização da fase de análise de requisitos do projeto, houve a necessidade de utilizar conceitos
com os quais os elementos do grupo não estavam familiarizados.
A sua utilização obrigou a uma aprendizagem ativa, por forma a se conseguir dar resposta aos objetivos propostos.
Tal como qualquer instância de aprendizagem, houve dificuldades na sua implementação.
Em particular: familiarização da linguagem UML e \underline{capacidade de abstração} \underline{do problema}.
Esta última foi a mais difícil de ultrapassar, pois o grupo tendia a olhar para o modelo como sendo a realidade
e não uma representação abstrata da mesma.
A título exemplificativo, inicialmente foi definido um \textit{use case} que refletia todo um fluxo real
(desde que o cliente entra até que sai com o problema resolvido).
Posteriormente, através de estudo e acompanhamento da equipa docente, foi possível melhorar esta capacidade de abstração
e atingir um resultado final que o grupo acredita ir de encontro com os objetivos propostos.
Não obstante, o grupo ter atravessado algumas dificuldades, os resultados de aprendizagem acabaram por compensar.
Com a concretização desta fase, o grupo encontra-se mais capaz de modelar um problema utilizando a abstração necessária.

Apesar do grupo considerar que a solução a que chegou é suficiente e adequada, a sua falta de experiência
ao nível da conceção e implementação de soluções não lhe permite ser capaz de fazer a devida análise.
Por este motivo, o grupo tem consciência que se encontra perante um processo iterativo e que
na fase seguinte (modelação conceptual e implementação da solução), provavelmente, encontrará algumas dificuldades
em que será obrigado a fazer alterações da fase a que o presente relatório se refere.


%==========================================================================
% BEGIN LISTA DE SIGLAS E ACRÓNIMOS
%==========================================================================

%% Portuguese babel does not translate this environment
\renewcommand{\nomname}{Lista de Siglas e Acrónimos}

%% Text that can be shown before acronyms list
\renewcommand{\nompreamble}{<<Apresentar uma lista com todas as siglas e acrónimos utilizados durante a realização do trabalho. O formato base para esta lista deverá ser da forma como abaixo se apresenta.>>}

%% acronyms
\nomenclature[01]{\textbf{NIF}}{Número de Idenficação Fiscal}
\nomenclature[02]{UC}{Unidade Curricular}
\nomenclature[03]{DSS}{Desenvolvimento de Sistemas de Software}
\nomenclature[04]{aplicação}{Sistema de Gestão para Centro de Reparação de Equipamento Eletrónico}

%% Show acronyms
\printnomenclature

%==========================================================================
% END LISTA DE SIGLAS E ACRÓNIMOS
%==========================================================================


%==========================================================================
% BEGIN ANEXOS
%==========================================================================

\addchap{Anexos}
\addsec{Anexo 1 - Cenários } \label{cenarios}
\subfile{scenarios/scenario_1.tex}
\subfile{scenarios/scenario_2.tex}
\subfile{scenarios/scenario_3.tex}
\subfile{scenarios/scenario_4.tex}
\subfile{scenarios/scenario_5.tex}

%==========================================================================
% END ANEXOS
%==========================================================================


\end{document}