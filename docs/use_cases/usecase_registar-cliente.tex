\documentclass[../relatorio.tex]{subfiles}
\begin{document}
Registo de um cliente no sistema, utilizando os respetivos dados.
\begin{itemize}
    \item[Use Case] {\underline{Registar cliente}}
    \item[Cenários] {1 e 2}
    \item[Pré-condição] {Colaborador autenticado}
    \item[Pós-condição] {Novo cliente fica registado no sistema}
          \begin{flushleft}
              \textbf{Fluxo Normal}
          \end{flushleft}
          \begin{enumerate}
              \item Funcionário insere identificador (NIF) do cliente
              \item Sistema verifica identificador
              \item Funcionário insere dados do cliente
              \item Sistema regista o cliente
          \end{enumerate}
          \begin{flushleft}
              \textbf{Fluxo de Exceção 1 (passo 2) [identificador já existe]}
          \end{flushleft}
          \begin{itemize}
              \item[2.1]{Sistema comunica que identificador já está registado}
          \end{itemize}
\end{itemize}
\begin{landscape}
    \begin{table}[!h]
        \centering
        \begin{tabular}{|p{5cm}|p{1cm}|p{4cm}|p{6cm}|p{3cm}|}
            \hline
            \rowcolor{gray!20!white}
            Use Case & Fluxo                                            & Responsabilidade & API & Subsistema \\
            \hline
            \rowcolor{yellow}
            Funcionário insere identificador (NIF) do cliente
                     & UI
                     & 
                     & 
                     & 
            \\
            \hline
            Sistema verifica identificador do cliente
                     & 
                     & verifica se cliente existe
                     & existeCliente(CliID : String) : Boolean
                     & SubUtilizadores
            \\
            \hline
            \rowcolor{yellow}
            Funcionário insere dados do cliente
                     & UI
                     & 
                     & 
                     & 
            \\
            \hline
            Sistema regista o cliente
                     & 
                     & regista cliente
                     & registaCliente(CliID : String, numero : String, email : String)
                     & SubUtilizadores
            \\
            \hline
            \rowcolor{red!30}
            Exceção  & 1                                                 &  \multicolumn{3}{c}{identificador já existe}\\
            \hline
            Sistema comunica que identificador já está registado
                     & UI
                     & 
                     & 
                     & 
            \\
            \hline
        \end{tabular}
        \caption{Análise Use Case - Registar Cliente (ver \ref{registar_cliente})}
    \end{table}
\end{landscape}
\end{document}