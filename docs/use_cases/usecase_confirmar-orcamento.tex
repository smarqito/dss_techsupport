\documentclass[../relatorio.tex]{subfiles}
\begin{document}
Registo da confirmação do orçamento, aquando da aceitação do cliente.

Em termos estruturais, o use case \textbf{Confirmar Orcamento} encontra-se semalhante ao seu estado anterior.
Foram adicionados alguns passos ao fluxo normal como o quinto e sexto, pois o método está preparado para 
criar uma reparação e a associar a um técnico, atualizando a sua agenda.
Foram alterados os métodos na análise de requisitos, pelo que a lógica toda é tratada pelo método \textit{AlteraEstadoOrc}.
\begin{itemize}
    \item[Use Case] {\underline{Confirmar orçamento}}
    \item[Cenários] {1 e 3}
    \item[Pré-condição] {Funcionário autenticado e orçamento concluído}
    \item[Pós-condição] {Adicionado pedido de reparação ao sistema}
          \begin{flushleft}
              \textbf{Fluxo Normal}
          \end{flushleft}
          \begin{enumerate}
              \item Funcionário insere o identificador do orçamento 
              \item Sistema devolve orçamento
              \item Funcionário regista orçamento como aceite
              \item Sistema atualiza orçamento para reparação 
              \item Sistema associa reparação a um Técnico e atualiza a sua agenda 
              \item Sistema insere reparação à lista de equipamentos por reparar
          \end{enumerate}
          \begin{flushleft}
              \textbf{Fluxo de exceção 1 (passo 3) [Cliente recusa reparação]}
          \end{flushleft}
          \begin{itemize}
              \item[3.1]{Funcionário regista orçamento como recusado}
              \item[3.2]{Sistema adiciona orçamento ao arquivo}
              \item[3.3]{Sistema coloca equipamento pronto a levantar}
          \end{itemize}
\end{itemize}
\begin{landscape}
    \begin{table}[!h]
        \centering
        \begin{tabular}{|p{5cm}|p{1cm}|p{4cm}|p{6cm}|p{3cm}|}
            \hline
            \rowcolor{gray!20!white}
            Use Case & Fluxo                                            & Responsabilidade & API & Subsistema \\
            \hline
            \rowcolor{yellow}
            Funcionário pede orçamento ao sistema
                     & UI
                     & 
                     & 
                     & 
            \\
            \hline
            Sistema devolve orçamento
                     & 
                     & obtém o orçamento
                     & getOrcamento(ref: String): Orcamento
                     & SSReparacoes
            \\
            \hline
            \rowcolor{yellow}
            Funcionário regista orçamento como aceite
                     & UI
                     & 
                     & 
                     & 
            \\
            \hline
            Sistema atualiza orçamento para reparação
            Sistema associa reparação a um Técnico e atualiza a sua agenda
            Sistema insere pedido de reparação à lista de equipamentos por reparar
                     & 
                     & Criar reparação e inseri-la no sistema, associando-a a um técnico
                     & alteraEstadoOrc(orcID: String, state: OrcamentoEstado)
                     & SSReparacoes
            \\
            \hline
            \rowcolor{yellow}
            Funcionário regista orçamento como recusado
                     & UI
                     & 
                     & 
                     & 
            \\
            \hline
            \rowcolor{red!30}
            Exceção  & 1                                                 &  \multicolumn{3}{c}{Cliente recusa reparação}\\
            \hline
            Sistema adiciona orçamento ao arquivo
            Sistema coloca equipamento disponível para entrega
                     & 
                     & muda o estado do orçamento para arquivado e o do equipamento associado para Pronto a Levantar
                     & alteraEstadoOrc(orcID: String, state: OrcamentoEstado)
                     & SSReparacoes
            \\
            \hline
        \end{tabular}
        \caption{Análise Use Case - Confirmar Orcamento (ver \ref{confirmar_orcamento})}
    \end{table}
\end{landscape}
\end{document}