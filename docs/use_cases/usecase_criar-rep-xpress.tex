\documentclass[../relatorio.tex]{subfiles}
\begin{document}
Registo de uma nova reparação expresso no sistema, incluindo o material utilizado, tempo consumido e preço.
\begin{itemize}
    \item[Use Case] {\underline{Criar Reparação Expresso}}
    \item[Cenários] {2}
    \item[Pré-condição] {Colaborador especializado autenticado}
    \item[Pós-condição] {Reparação expresso fica registada no Sistema}
          \begin{flushleft}
              \textbf{Fluxo Normal}
          \end{flushleft}
          \begin{enumerate}
              \item Colaborador especializado insere nome da reparação
              \item Sistema verifica nome da reparação
              \item Colaborador especializado insere dados da reparacao % descrição, ...
              \item Colaborador especializado define plano de trabalhos
              \item Sistema regista reparação expresso
          \end{enumerate}
          \begin{flushleft}
            \textbf{Fluxo de Exceção 1 (passo 2) [Nome da reparação já existe]}
        \end{flushleft}
        \begin{itemize}
            \item[2.1] Sistema comunica que nome da reparação já existe
        \end{itemize}
\end{itemize}
\begin{landscape}
    \begin{table}[!h]
        \centering
        \begin{tabular}{|p{5cm}|p{1cm}|p{4cm}|p{6cm}|p{3cm}|}
            \hline
            \rowcolor{gray!20!white}
            Use Case & Fluxo                                            & Responsabilidade & API & Subsistema \\
            \hline
            \rowcolor{yellow}
            Colaborador especializado insere nome da reparação
                     & UI
                     & 
                     & 
                     & 
            \\
            \hline
            Sistema verifica nome da reparação
                     & 
                     & verifica se o nome já existe
                     & existeRepXpresso(nome : String) : Boolean
                     & SubReparacoes
            \\
            \hline
            \rowcolor{yellow}
            Colaborador especializado insere dados da reparacao
                     & UI
                     & 
                     & 
                     & 
            \\
            \hline
            Colaborador especializado define plano de trabalhos
                     & UI
                     & 
                     & 
                     & 
            \\
            \hline
            Sistema regista reparação expresso
                     & 
                     & regista reparação expresso
                     & registaRepXpresso(nome : String, desc : String, preco : Double, tempo : Integer)
                     & SubReparacoes
            \\
            \hline
            \rowcolor{red!30}
            Exceção  & 1                                                 &  \multicolumn{3}{c}{Nome da reparação já existe}\\
            \hline
            Sistema comunica que nome da reparação já existe
                     & UI
                     & 
                     & 
                     & 
            \\
            \hline
        \end{tabular}
        \caption{Análise Use Case - Criar Reparação Expresso (ver \ref{criar_rep_expresso})}
    \end{table}
\end{landscape}
\end{document}