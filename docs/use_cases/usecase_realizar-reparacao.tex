\documentclass[../relatorio.tex]{subfiles}
\begin{document}
Reparação de um equipamento, realizada por um técnico, tendo em conta o plano de trabalhos.
\begin{itemize}
    \item[Use Case] {\underline{Realizar reparação}}
    \item[Cenários] {4}
    \item[Pré-condição] {Técnico autenticado e equipamentos para reparar}
    \item[Pós-condição] {Equipamento reparado}
          \begin{flushleft}
              \textbf{Fluxo Normal}
          \end{flushleft}
          \begin{enumerate}
              \item Técnico consulta a sua agenda e indica o identificador da reparação a ser efetuada
              \item Sistema devolve reparação
              \item Sistema altera estado da reparação para "emReparacao"
              \item Técnico pede o plano de trabalhos da reparação
              \item Técnico indica os passos da reparação concretizados
              \item Sistema regista passos da reparação como realizados
              \item Sistema regista a conclusão da reparação
              \item Sistema muda estado da reparação para reparada
              \item Técnico coloca equipamento disponível para levantamento
              \item Sistema muda estado de equipamento para pronto para levantamento
              \item Sistema regista data e hora
          \end{enumerate}
          \begin{flushleft}
              \textbf{Fluxo de Exceção 1 (passo 7) [falta de tempo]}
          \end{flushleft}
          \begin{itemize}
              \item[3.1]{Técnico informa motivo de interrupção}
              \item[3.2]{Sistema altera estado da reparação para "aguardaReparacao"}
          \end{itemize}
          \begin{flushleft}
              \textbf{Fluxo de Exceção 2 (passo 7) [falta de peças]}
          \end{flushleft}
          \begin{itemize}
              \item[3.1]{Técnico informa motivo de interrupção}
              \item[3.2]{Sistema altera estado da reparação para "aguardaReparacao"}
          \end{itemize}
          \begin{flushleft}
              \textbf{Fluxo Alternativo 3 (passo 7) [reparação excede valor orçamentado]}
          \end{flushleft}
          \begin{itemize}
              \item[3.1]{Técnico insere contacto com cliente}
              \item[3.2]{Sistema regista data, hora e técnico associado}
              \item[3.3]{Regressa ao passo 3}
          \end{itemize}
          \begin{flushleft}
              \textbf{Fluxo de Exceção 4 (passo 3.2, fluxo 2) [cliente recusa continuidade da reparação]}
          \end{flushleft}
          \begin{itemize}
              \item[3.2.1]{Sistema coloca reparação como cancelada}
              \item[3.2.2]{Técnico repõe o equipamento}
              \item[3.2.3]{Técnico regista as horas gastas}
              \item[3.2.4]{Técnico coloca equipamento disponível para levantamento}
              \item[3.2.5]{Sistema muda estado de equipamento para pronto para levantamento}
              \item[3.2.6]{Sistema muda estado da reparação para cancelada}
              \item[3.2.7]{Sistema regista data e hora} 
          \end{itemize}
\end{itemize}

Como a adição das agendas, já não é viável pedir a reparação mais urgente, uma vez que todas as reparações ja foram distribuidas
uniformemente pelos diferentes técnicos.
Outra modificação efetuda neste \textit{Use Case} foi a alteração do estado da reparação aquando do ínicio da reparação para "emReparacao".

Na primeira fase estava estabelecido que quando uma reparação era interrompida, esta era adicionada à lista de reparações pendentes, porém como
referido anteriormente esta lista deixou de existir com o uso das agendas e a solução encontrada foi alterar o estado da reparação para "aguardaReparacao".
Apesar de se verificar esta alteração ao nível do \textit{Use Case}, o mesmo não acontece ao nível da implementação que não foi estabelecido nenhum
mecanismo capaz de lidar com interrupções.

\begin{landscape}
    \begin{table}[!h]
        \centering
        \begin{tabular}{|p{5cm}|p{1cm}|p{4cm}|p{6cm}|p{3cm}|}
            \hline
            \rowcolor{gray!20!white}
            Use Case & Fluxo                                            & Responsabilidade & API & Subsistema \\
            \hline
            \rowcolor{yellow}
            Técnico consulta a sua agenda e indica o identificador da reparação a ser efetuada
                     & UI
                     & 
                     & 
                     & 
            \\
            \hline
            Sistema devolve reparação
                     & 
                     & seleciona a reparação indicada
                     & getReparacao(repId : String) : Reparacao
                     & SubReparacoes
            \\
            \hline
            Sistema altera estado da reparação para emReparacao
                     & 
                     & muda estado de reparação para emReparacao
                     & alterarEstadoRep(repID : String , estado : ReparacaoEstado)
                     & SubReparacoes
            \\
            \hline
            \rowcolor{yellow}
            Técnico pede o plano de trabalhos da reparação
                     & UI
                     & 
                     & 
                     & 
            \\
            \hline
            \rowcolor{yellow}
            Técnico indica os passos da reparação concretizados
                     & UI
                     & 
                     & 
                     & 
            \\
            \hline
            Sistema regista passos da reparação como realizados
                     & 
                     & regista os passos da reparação realizados 
                     & registaPassosRep(repID : String, passosRealizados : List<PassoReparacao>)
                     & SubReparacoes
            \\
            \hline
            Sistema regista a conclusão da reparação
            Sistema muda estado da reparação para reparada
                     & 
                     & regista reparação como concluida
                     & alterarEstadoRep(repID : String, estado : EstadoReparacao)
                     & SubReparacoes
            \\
            \hline
            Técnico coloca equipamento disponível para levantamento
                     & UI
                     & 
                     & 
                     & 
            \\
            \hline
            Sistema muda estado de equipamento para pronto para levantamento
            Sistema regista data e hora
                     & 
                     & muda estado de equipamento para pronto para levantamento e regista hora
                     & alteraEstadoEq(equiID : String, state : EstadoEquipamento)
                     & SubClientes
            \\
            \hline
            \rowcolor{red!30}
            Exceção  & 1                                                 &  \multicolumn{3}{c}{falta de tempo}\\
            \hline
            \rowcolor{yellow}
            Técnico informa motivo de interrupção
                     & UI
                     & 
                     & 
                     & 
            \\
            \hline
            Sistema altera estado da reparação para aguardaReparacao
                     & 
                     & muda estado de reparação para aguardaReparacao
                     & alterarEstadoRep(repID : String , estado : ReparacaoEstado)
                     & SubReparacoes
            \\
            \hline
            \rowcolor{red!30}
            Exceção  & 2                                                 &  \multicolumn{3}{c}{falta de peças}\\
            \hline
            \rowcolor{yellow}
            Técnico informa motivo de interrupção
                     & UI
                     & 
                     & 
                     & 
            \\
            \hline
            Sistema altera estado da reparação para aguardaReparacao
                     & 
                     & muda estado de reparação para aguardaReparacao
                     & alterarEstadoRep(repID : String , estado : ReparacaoEstado)
                     & SubReparacoes
            \\
            \hline
            \rowcolor{green!30}
            Alternativo  & 3                                                 &  \multicolumn{3}{c}{reparação excede valor orçamentado}\\
            \hline
            \rowcolor{yellow}
            Técnico insere contacto com cliente
                     & UI
                     & 
                     & 
                     & 
            \\
            \hline
            Sistema regista data, hora e técnico associado
                     & 
                     & regista a data, hora e técnico associado ao contacto ao cliente 
                     & registaContacto(msg : String, tecID : String)
                     & SubReparacoes
            \\
            \hline
            Regressa ao passo 3
                     & 
                     & 
                     & 
                     & 
            \\
            \hline
            \rowcolor{red!30}
            Exceção  & 4                                               &  \multicolumn{3}{c}{cliente recusa continuidade da reparação}\\
            \hline
            Sistema coloca reparação como cancelada
                     & 
                     & cancela pedido reparação
                     & alterarEstadoRep(repID : String, estado : EstadoReparacao)
                     & SubReparacoes
            \\
            \hline
            Técnico repõe o equipamento
                     & 
                     & 
                     & 
                     & 
            \\
            \hline
            \rowcolor{yellow}
            Técnico regista as horas gastas
                     & UI
                     & 
                     & 
                     & 
            \\
            \hline
            \rowcolor{yellow}
            Técnico coloca equipamento disponível para levantamento
                     & UI
                     & 
                     & 
                     & 
            \\
            \hline
            Sistema muda estado de equipamento para pronto para levantamento
            Sistema regista data e hora
                     & 
                     & muda estado de equipamento para pronto para levantamento e regista hora
                     & alteraEstadoEq(equiID : String, state : EstadoEquipamento)
                     & SubClientes
            \\
            \hline
            Sistema muda estado da reparação para cancelada
            Sistema regista data e hora
                     & 
                     & muda estado da reparação para cancelada
                     & alterarEstadoRep(repID : String , estado : ReparacaoEstado)
                     & SubReparacoes
            \\
            \hline
        \end{tabular}
        \caption{Análise Use Case - Realizar Reparação (ver \ref{realizar_rep})}
    \end{table}
\end{landscape}
\end{document}