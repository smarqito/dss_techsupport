\documentclass[../relatorio.tex]{subfiles}
\begin{document}
Progressão de uma reparação, através da realização do passo.
É atualizado o \textit{stock} do material gasto e registado do tempo efetivamente consumido.
\begin{itemize}
    \item[Use Case] {\underline{Realizar passo da reparação}}
    \item[Cenários] {4}
    \item[Pré-condição] {Técnico autenticado e material necessário disponível}
    \item[Pós-condição] {Equipamento progride na sua reparação}
          \begin{flushleft}
              \textbf{Fluxo Normal}
          \end{flushleft}
          \begin{enumerate}
              \item Técnico indica referência do material gasto
              \item Sistema altera \textit{stock} do material gasto
              \item Técnico indica tempo e custo efetivamente gasto 
              \item Sistema regista tempo e custo efetivamente gasto e Técnico que o realizou
          \end{enumerate}
\end{itemize}

\begin{landscape}
    \begin{table}[!h]
        \centering
        \begin{tabular}{|p{5cm}|p{1cm}|p{4cm}|p{6cm}|p{3cm}|}
            \hline
            \rowcolor{gray!20!white}
            Use Case & Fluxo & Responsabilidade & API & Subsistema \\
            \hline
            \rowcolor{yellow}
            Técnico indica referência do material gasto
                     & UI
                     &
                     &
                     &
            \\
            \hline
            Sistema altera \textit{stock} do material gasto
                     & 
                     & Sistema atualiza o número de \textit{stock} do material disponível
                     & removeStock(matID: String)
                     & SubReparacoes
            \\
            \hline
            \rowcolor{yellow}
            Técnico indica tempo efetivamente gasto
                     & UI
                     &
                     &
                     &
            \\
            \hline
            Sistema regista tempo e custo efetivamente gasto e Técnico que o realizou
                     &
                     & Sistema regista tempo efetivamente gasto
                     & registaPasso(repID: String, passoID: String, mins: Integer, custo: Double, tecID: String)
                     & SubReparacoes
            \\
            \hline
        \end{tabular}
        \caption{Análise Use Case - Realizar passo da reparação (Ver \ref{realizar_passo_rep})}
    \end{table}
\end{landscape}

\end{document}