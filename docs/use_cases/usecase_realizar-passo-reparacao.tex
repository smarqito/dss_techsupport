\documentclass[../relatorio.tex]{subfiles}
\begin{document}
Progressão de uma reparação, através da realização do passo.
É atualizado o \textit{stock} do material gasto e registado do tempo efetivamente consumido.

De maneira a não aumentar a complexidade do trabalho não foram considerados stocks de materiais, apesar de estes serem referidos neste \textit{Use Case}.
Assim quando se realiza o passo não é necessário decrementar o stock em função da quantidade de material gasto.
Como referido anteriormente, a adição da agenda permitiu que as reparações ficassem logo atribuidas a um técnico no momento em que foram criadas. Assim
já não é possível uma reparação ser realizada por mais que um técnico e ao contrário do que foi estabelecido na primeira fase já não é necessário identificar 
o autor de cada passo.
\begin{itemize}
    \item[Use Case] {\underline{}}
    \item[Cenários] {4}
    \item[Pré-condição] {Técnico autenticado e material necessário disponível}
    \item[Pós-condição] {Passo adicionado à lista de passos realizados}
          \begin{flushleft}
              \textbf{Fluxo Normal}
          \end{flushleft}
          \begin{enumerate}
              \item Técnico indica a identificação da reparação a ser efetuada 
              \item Sistema seleciona a reparação
              \item Sistema seleciona o passo de reparação atual
              \item Técnico indica referência do material gasto
              \item Sistema altera \textit{stock} do material gasto
              \item Técnico indica tempo e custo efetivamente gasto 
              \item Sistema regista tempo e custo efetivamente gasto
              \item Sistema adiciona passo à lista de passos realizados
          \end{enumerate}
\end{itemize}

\begin{landscape}
    \begin{table}[!h]
        \centering
        \begin{tabular}{|p{5cm}|p{1cm}|p{4cm}|p{6cm}|p{3cm}|}
            \hline
            \rowcolor{gray!20!white}
            Use Case & Fluxo & Responsabilidade & API & Subsistema \\
            \hline
            \rowcolor{yellow}
            Técnico indica a identificação da reparação a ser efetuada 
                     & UI
                     &
                     &
                     &
            \\
            \hline
            Sistema seleciona a reparação
                     & 
                     & Sistema seleciona a reparação a partir do seu ID
                     & getRep(repID: String): Reparacao
                     & SubReparacoes
            \\
            \hline
            Sistema seleciona o passo de reparação atual
                    & 
                    & Sistema seleciona o passo de reparação a partir do seu ID
                    & getPassoAtual(): PassoReparacao
                    & SubReparacoes
            \\
            \hline
            \rowcolor{yellow}
            Técnico indica referência do material gasto
                     & UI
                     &
                     &
                     &        
            \\
            \hline
            \rowcolor{yellow}
            Sistema altera \textit{stock} do material gasto
                     & 
                     & Sistema decrementa o \textit{stock} do material gasto
                     & consumeMaterial(mat: Material)
                     & SubReparacoes
            \\
            \hline
            \rowcolor{yellow}
            Técnico indica tempo e custo efetivamente gasto
                     & UI
                     & 
                     &
                     & 
            \\
            \hline
            Sistema regista tempo e custo efetivamente gasto e Técnico
            Sistema adiciona passo à lista de passos realizados
                     & 
                     & Sistema regista os valores do passo após ter sido concretizado
                     & registaPassoRealizado(tempo : Integer, custo : Double)
                     & SubReparacoes
            \\
            \hline
        \end{tabular}
        \caption{Análise Use Case - Realizar passo da reparação (Ver \ref{realizar_passo_rep})}
    \end{table}
\end{landscape}

\end{document}