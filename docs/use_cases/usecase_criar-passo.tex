\documentclass[../relatorio.tex]{subfiles}
\begin{document}
Registo no sistema de um novo passo, introduzindo o tempo e peças necessárias para a sua concretização.
\begin{itemize}
    \item[Use Case] {\underline{Criar passo de reparação}}
    \item[Cenários] {3}
    \item[Pré-condição] {Colaborador especializado autenticado}
    \item[Pós-condição] {Passo de reparação é adicionado ao Sistema}
          \begin{flushleft}
              \textbf{Fluxo Normal}
          \end{flushleft}
          \begin{enumerate}
              \item Colaborador especializado insere nome do passo da reparação
              \item Sistema verifica nome do passo da reparação
              \item Colaborador especializado insere dados do passo da reparação
              \item Colaborador especializado insere categoria do material
              \item Sistema verifica categoria do material
              \item Sistema regista passo de reparação
          \end{enumerate}
          \begin{flushleft}
            \textbf{Fluxo de Exceção 1 (passo 2) [Nome já existe]}
        \end{flushleft}
        \begin{itemize}
            \item[2.1]  Sistema comunica que nome do passo da reparação já existe
        \end{itemize}
          \begin{flushleft}
		      \textbf{Fluxo Alternativo 2 (passo 4) [Material não registado]}
	      \end{flushleft}
	      \begin{itemize}
		      \item[4.1]  $\ll include \gg$ Adicionar Material
              \item[4.2] Regressa ao passo 5
	      \end{itemize}
          \begin{flushleft}
            \textbf{Fluxo Alternativo 3 (passo 5) [Categoria não existe]}
        \end{flushleft}
        \begin{itemize}
            \item[5.1] Sistema comunica que categoria não existe 
            \item[5.2] Colaborador especializado insere nome da categoria 
            \item[5.3] Sistema regista nova categoria
            \item[5.4] Regressa ao passo 6
        \end{itemize}
\end{itemize}
\end{document}