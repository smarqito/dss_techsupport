\documentclass[../relatorio.tex]{subfiles}
\begin{document}
Registo de um orçamento relativo a um pedido de orçamento, através do seu identificador.
O técnico constrói o plano de trabalhos e, quando finalizado, o sistema envia orçamento ao cliente,
registando data e hora.
\begin{itemize}
    \item[Use Case] {\underline{Fazer orçamento}}
    \item[Cenários] {3}
    \item[Pré-condição] {Técnico autenticado e existe pedidos de orçamento}
    \item[Pós-condição] {Orçamento gerado, registado e comunicado ao cliente}
          \begin{flushleft}
              \textbf{Fluxo Normal}
          \end{flushleft}
          \begin{enumerate}
              \item Técnico solicita pedido de orçamento mais antigo
              \item Sistema devolve pedido de orçamento mais antigo
              \item Técnico utiliza identificador para levantar equipamento
              \item Técnico define plano de trabalhos para a reparação
              \item Sistema regista plano de trabalhos para a reparação
              \item Sistema gera orçamento e regista-o
              \item Sistema envia orçamento ao cliente
              \item Sistema regista data e hora de comunicação
          \end{enumerate}
          \begin{flushleft}
              \textbf{Fluxo de Exceção 1 (passo 4) [Equipamento não pode ser reparado]}
          \end{flushleft}
          \begin{itemize}
              \item[4.1] {Técnico regista que não é possível fazer reparação}
              \item[4.2] {Sistema comunica o cliente}
          \end{itemize}
\end{itemize}
\begin{landscape}
    \begin{table}[!h]
        \centering
        \begin{tabular}{|p{5cm}|p{1cm}|p{4cm}|p{6cm}|p{3cm}|}
            \hline
            \rowcolor{gray!20!white}
            Use Case & Fluxo & Responsabilidade & API & Subsistema \\
            \hline
            \rowcolor{yellow}
            Técnico solicita pedido de orçamento mais antigo
                     & 
            UI
                     & 
                     & 
                     & 
            \\
            \hline
            Sistema devolve pedido de orçamento mais antigo
                     & 
                     & 
            Calcula o pedido de orçamento de orçamento há mais tempo no sistema
                     & 
            getOlderOrcamento()
                     & 
            SubReparacoes
            \\
            \hline
            \rowcolor{yellow}
            Técnico define plano de trabalhos para a reparação
                     & 
            UI 
                     & 
                     & 
                     & 
            \\
            \hline
            Sistema regista plano de trabalhos para a reparação
                     & 
                     & 
            associa plano de trabalho ao orçamento
                     & 
            registaPT(orcId: String, passos: List<PassoReparacao>)
                     & 
            SubReparacoes
            \\
            \hline
            Sistema gera orçamento e regista-o
                     & 
                     & 
            Cria um documento com o resumo do orçamento (plano de trabalho + preço)
                     & 
            generateOrc(orcId: String)
                     & 
            SubReparacoes
            \\
            \hline
            Sistema envia orçamento ao cliente
            Sistema regista data e hora de comunicação
                     & 
                     & 
            envia orçamento ao cliente com o contacto na ficha de orçamento
                     & 
            enviarOrcamento(orcId: String)
                     & 
            SubReparacoes
            \\
            \hline
            \rowcolor{red!30}
            Exceção
                     & 
            1 
                     & 
            \multicolumn{3}{c}{Equipamento não pode ser reparado}
            \\
            \rowcolor{yellow}
            \hline
            Técnico regista que não é possível fazer reparação
                     & 
            UI
                     & 
                     & 
                     & 
            \\
            \hline
            Sistema comunica o cliente
                     & 
                     & 
            envia mensagem ao cliente que há erros
                     & 
            comunicarErro(orcId: String, msg: String)
                     & 
            SubReparacoes
            \\
            \hline
        \end{tabular}
        \caption{Análise Use Case - Fazer orçamento (ver \ref{fazer_orcamento})}
    \end{table}
\end{landscape}

\end{document}