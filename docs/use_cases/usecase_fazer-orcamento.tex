\documentclass[../relatorio.tex]{subfiles}
\begin{document}
\begin{itemize}
    \item[Use Case] {\underline{Fazer orçamento}}
    \item[Cenários] {3}
    \item[Pré-condição] {Técnico autenticado e existe pedidos de orçamento}
    \item[Pós-condição] {Orçamento gerado, registado e comunicado ao cliente}
          \begin{flushleft}
              \textbf{Fluxo Normal}
          \end{flushleft}
          \begin{enumerate}
              \item Técnico pede orçamento mais antigo
              \item Sistema devolve orçamento mais antigo
              \item Técnico utiliza identificador para levantar equipamento
              \item Técnico regista plano de trabalhos para a reparação
                    % definir plano de trabalhos???
              \item Sistema gera orçamento e regista-o
              \item Sistema envia orçamento ao cliente
          \end{enumerate}
          \begin{flushleft}
              \textbf{Fluxo de Exceção 1 (passo 4) [Equipamento não pode ser reparado]}
          \end{flushleft}
          \begin{itemize}
              \item[4.1] {Técnico regista que não é possível fazer reparação}
              \item[4.2] {Sistema comunica o cliente}
          \end{itemize}
\end{itemize}
\end{document}