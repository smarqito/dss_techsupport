\documentclass[../relatorio.tex]{subfiles}
\begin{document}
Arquivamento dos orçamentos que não obtiveram resposta por parte do cliente há mais de 30 dias.
Corresponde a um fluxo automático que deverá ser executado diariamente.
\begin{itemize}
    \item[Use Case] {\underline{Arquivar orçamento}}
    \item[Cenários] {1}
    \item[Pré-condição] {System timer ativo}
    \item[Pós-condição] {Orçamentos com mais de 30 dias arquivados}
          \begin{flushleft}
              \textbf{Fluxo Normal}
          \end{flushleft}
          \begin{enumerate}
              \item Gestor pede para arquivar os orçamentos que ja passaram 30 dias desde que foram enviados
              \item Sistema filtra todos os orçamentos enviados
              \item Sistema filtra orçamentos por data, ficando com os superiores a 30 dias
              \item Sistema adiciona orçamentos ao arquivo e apresenta-os
          \end{enumerate}
\end{itemize}


\begin{landscape}
    \begin{table}[!h]
        \centering
        \begin{tabular}{|p{5cm}|p{1cm}|p{4cm}|p{6cm}|p{3cm}|}
            \hline
            \rowcolor{gray!20!white}
            Use Case & Fluxo    & Responsabilidade & API & Subsistema    \\
            \hline
            \rowcolor{yellow}
            Gestor pede para arquivar os orçamentos que ja passaram 30 dias desde que foram enviados
                    & UI 
                    &                                        
                    &                                                          
                    &               
            \\
            \hline
            Sistema filtra todos os orçamentos enviados
            Sistema filtra orçamentos por data, ficando com os superiores a 30 dias
                    &          
                    & filtrar orçamentos com mais de 30 dias 
                    & filterOrcamentos(p : (o -> enviado() \& passouPrazo())): List<Orcamento> 
                    & SSReparacoes 
            \\
            \hline
            Sistema adiciona orçamentos ao arquivo                                  
                    &          
                    & mudar o estado do orçamento    
                    & arquivaOrcamentos():List<Orcamento>                    
                    & SSReparações 
            \\
            \hline
        \end{tabular}
        \caption{Identidade do Projeto.}
    \end{table}
\end{landscape}

\end{document}