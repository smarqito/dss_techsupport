\documentclass[../relatorio.tex]{subfiles}
\begin{document}
Registo de um novo colaborador no sistema pelo Gestor, sendo-lhe associado um número de identificação.
\begin{itemize}
    \item[Use Case] {\underline{Registar Colaborador}}
    \item[Cenários] {1, 2, 3, 4 e 5} 
    \item[Pré-condição] {Gestor autenticado}
    \item[Pós-condição] {Novo colaborador registado}
            \begin{flushleft}
                \textbf{Fluxo Normal}
            \end{flushleft}
            \begin{enumerate}
                \item Gestor insere nome e função do Colaborador
                \item Sistema calcula número de identificação
                \item Sistema regista Colaborador
            \end{enumerate}
\end{itemize}

\begin{landscape}
    \begin{table}[!h]
        \centering
        \begin{tabular}{|p{5cm}|p{1cm}|p{4cm}|p{6cm}|p{3cm}|}
            \hline
            \rowcolor{gray!20!white}
            Use Case & Fluxo                                            & Responsabilidade & API & Subsistema \\
            \hline
            \rowcolor{yellow}
            Gestor insere nome e função do Colaborador
                     & UI
                     & 
                     & 
                     & 
            \\
            \hline
            Sistema calcula número de identificação
                     & 
                     & calcula número de identificação do novo colaborador
                     & GetCurrentID() : Integer
                     & SubColaboradores
            \\
            \hline
            Sistema regista Colaborador
                     & 
                     & regista colaborador
                     & registaColaborador(nome : String, tipo : Class<?>) : String
                     & SubColaboradores
            \\
            \hline
        \end{tabular}
        \caption{Análise Use Case - Registar Colaborador (ver \ref{registar_colab})}
    \end{table}
\end{landscape}
\end{document}