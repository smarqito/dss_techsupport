\documentclass[../relatorio.tex]{subfiles}
\begin{document}
Resgisto do pedido de orçamento de um equipamento através do seu identificador.
\begin{itemize}
	\item[Use Case] {\underline{Pedir orçamento}}
	\item[Cenários] {1 e 2}
	\item[Pré-condição] {Funcionário autenticado e cliente registado}
	\item[Pós-condição] {Pedido de orçamento adicionado}
	      \begin{flushleft}
		      \textbf{Fluxo Normal}
	      \end{flushleft}
	      \begin{enumerate}
		      \item Funcionário insere identificador do cliente
              \item Sistema seleciona cliente
		      \item Funcionário insere identificador do equipamento
		      \item Sistema verifica identificador do equipamento
		      \item Funcionário insere descrição da avaria regista pedido de orçamento
		      \item Sistema adiciona pedido de orçamento
	      \end{enumerate}

	      \begin{flushleft}
		      \textbf{Fluxo Alternativo 1 (passo 4) [Equipamento não registado]}
	      \end{flushleft}
	      \begin{itemize}
		      \item[2.1] $\ll include \gg$ Registar equipamento
		      \item[2.2] Regressa ao passo 3
	      \end{itemize}
\end{itemize}
\begin{landscape}
    \begin{table}[!h]
        \centering
        \begin{tabular}{|p{5cm}|p{1cm}|p{4cm}|p{6cm}|p{3cm}|}
            \hline
            \rowcolor{gray!20!white}
            Use Case & Fluxo                                            & Responsabilidade & API & Subsistema \\
            \hline
            \rowcolor{yellow}
            Funcionário insere identificador do cliente
                     & UI
                     & 
                     & 
                     & 
            \\
            \hline
            Sistema seleciona cliente
                     & 
                     & procura cliente
                     & getCliente(CliID : String) : Cliente
                     & SubUtilizadores
            \\
            \hline
            \rowcolor{yellow}
            Funcionário insere identificador do equipamento
                     & UI
                     & 
                     & 
                     & 
            \\
            \hline
            Sistema verifica identificador do equipamento
                     & 
                     & verifica se o equipamento existe
                     & existeEquipamento(codR : String, marca: String) : Boolean
                     & SubReparacoes
            \\
            \hline
			\rowcolor{yellow}
			Funcionário regista pedido de orçamento
                     & UI
                     & 
                     & 
                     & 
            \\
            \hline
			Sistema adiciona pedido de orçamento
                     & 
                     & regista pedido de orçamento
                     & registarOrcamento(clienteID : String, equipID : String, descr: String)
                     & SubReparacoes
            \\
            \hline
            \rowcolor{green!30}
            Alternativo  & 1                                               &  \multicolumn{3}{c}{Equipamento não registado}\\
            \hline
            $\ll include \gg$ Registar equipamento
                     & 
                     & 
                     & 
                     & 
            \\
            \hline
			Regressa ao passo 3
                     & 
                     & 
                     & 
                     & 
            \\
            \hline
        \end{tabular}
        \caption{Análise Use Case - Pedir orçamento (ver \ref{pedir_orcamento})}
    \end{table}
\end{landscape}
\end{document}
