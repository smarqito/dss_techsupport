\documentclass[../relatorio.tex]{subfiles}
\begin{document}
Autenticação de um colaborador no sistema, validando o seu número de identificação.
\begin{itemize}
    \item[Use Case] {\underline{Autenticar Colaborador}}
    \item[Cenários] {1, 2, 3, 4 e 5}
    \item[Pré-condição] {Colaborador registado}
    \item[Pós-condição] {Colaborador autenticado}
          \begin{flushleft}
              \textbf{Fluxo Normal}
          \end{flushleft}
          \begin{enumerate}
              \item Colaborador insere número de identificação
              \item Sistema valida número de idenficação
          \end{enumerate}
\end{itemize}
\begin{landscape}
    \begin{table}[!h]
        \centering
        \begin{tabular}{|p{5cm}|p{1cm}|p{4cm}|p{6cm}|p{3cm}|}
            \hline
            \rowcolor{gray!20!white}
            Use Case         & 
            Fluxo            & 
            Responsabilidade & 
            API              & 
            Subsistema
            \\
            \hline
            \rowcolor{yellow}
            Colaborador insere número de identificação
                             & UI
                             & 
                             & 
                             & 
            \\
            \hline
            Sistema valida número de idenficação
                             & 
                             & validar identificação do colaborador
                             & validaIdentificacao(cod: String): Boolean
                             & SubUtilizadores
            \\
            \hline
        \end{tabular}
        \caption{Análise Use Case - Autenticar Colaborador (ver \ref{autenticar_colab})}
    \end{table}
\end{landscape}

\end{document}