\documentclass[../relatorio.tex]{subfiles}
\begin{document}
\textbf{Cenário 3} \label{cenario_3}

O técnico de reparações acede à lista de pedidos de orçamento e escolhe o mais antigo. Utiliza o código de registo do equipamento para o ir buscar
ao armazém e, depois de analisar a descrição do problema e o próprio equipamento, regista o plano de trabalhos para a reparação.

O plano de trabalhos consiste na sequência de passos necessária para efectuar a reparação. Para cada passo o técnico define uma previsão do tempo
necessário para a sua execução, bem como o custo das peças a utilizar, caso
sejam necessárias. Um passo pode ser decomposto em sub-passos, caso em
que a sua duração e custo de peças serão a soma das durações e custos de
peças dos sub-passos.

A definição do plano de trabalhos permite obter uma previsão do número total de horas de trabalho e o custo das peças necessárias. Com base nessa
informação, é criado um orçamento que é enviado ao cliente. Nesse orçamento vai também o prazo máximo de execução da reparação, calculado
em função do tempo necessário para reparar o equipamento e o trabalho
actualmente por realizar.


Variantes:
\begin{enumerate}
    \item Se o equipamento não poder ser reparado, essa informação é enviada ao cliente.
\end{enumerate}
\end{document}